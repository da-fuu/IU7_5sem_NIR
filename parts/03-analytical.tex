\section{Анализ предметной области}

\subsection[Понятие визуализации графов и основные задачи]{Понятие визуализации графов и основные \\ задачи}

Граф $G = (V, E)$, где $V$ --- множество вершин (узлов), а $E$ --- множество ребер (связей), является фундаментальной абстракцией для моделирования систем со сложной структурой взаимодействий. Задача визуализации графа заключается в построении отображения $M: V \rightarrow \mathbb{R}^k$ (обычно $k=2$ или $k=3$), которое сопоставляет каждому объекту координату в визуальном пространстве, и отображения ребер в виде геометрических примитивов (линий, кривых), соединяющих соответствующие вершины~\cite{kasyanov2015}.

В контексте <<больших данных>> (Big Data) под графами большой размерности обычно понимают структуры, содержащие от $10^4$ до $10^8$ элементов~\cite{vonLandesberger2011}. При работе с такими объемами классическая задача «рисования графов» трансформируется в задачу визуальной аналитики, где ключевыми становятся не столько критерии читаемости (количество пересечений ребер графа, площадь конечного изображения, количество изгибов ребер, наименьший угол между ребрами), сколько сохранение топологических свойств и выявление глобальной структуры данных (кластеров, аномалий)~\cite{naymushin2018heuristic}.

Основной вызов при визуализации больших графов заключается в противоречии между ограниченным разрешением устройств вывода (экранов) и когнитивными способностями человека с одной стороны, и объемом отображаемой информации с другой. Это приводит к феномену «визуального шума» или эффекту «волосяного шара», когда плотное переплетение ребер делает невозможным извлечение полезной информации без применения алгоритмов укладки и агрегации.

\subsection{Классификация подходов к пространственной укладке}

Пространственная укладка --- это процесс назначения координат вершинам графа. Для больших графов алгоритмы укладки принято классифицировать не по конкретным реализациям, а по математическим принципам, лежащим в их основе~\cite{kwon2018}.

\subsubsection{Силовые подходы}
Данный класс алгоритмов моделирует граф как физическую систему. Вершины рассматриваются как материальные тела, на которые действуют силы отталкивания (например, электростатические), а ребра — как пружины, создающие силы притяжения. Целью алгоритма является поиск равновесного состояния системы, соответствующего минимуму потенциальной энергии.
Достоинством подхода является способность выявлять симметрии и локальные кластерные структуры без предварительного обучения. Однако наивная реализация расчета сил требует вычислений сложности $O(|V|^2)$, что делает их неприменимыми для больших графов без использования техник пространственной аппроксимации.

\subsubsection{Спектральные подходы}
Спектральные методы базируются на линейной алгебре, в частности, на разложении матриц, ассоциированных с графом (матрицы смежности, лапласиана). Координаты вершин получаются из собственных векторов, соответствующих определенным собственным числам этих матриц.
Эти методы отличаются высокой скоростью работы и детерминированностью результата. Они эффективны для выявления глобальной структуры графа, однако часто дают результаты с высокой плотностью перекрытия вершин, что затрудняет интерактивную селекцию отдельных элементов.

\subsubsection{Методы снижения размерности}
Если вершины графа обладают многомерными векторными атрибутами, задача визуализации сводится к проекции из многомерного пространства $\mathbb{R}^d$ в пространство визуализации $\mathbb{R}^2$. Методы делятся на линейные (сохраняющие глобальные расстояния) и нелинейные (сохраняющие локальное соседство и топологию многообразия). Для больших графов критически важно использование алгоритмов, способных моделировать вероятностное распределение соседей, избегая проблемы «скученности».

\subsubsection{Многоуровневые подходы}
Для преодоления проблемы локальных минимумов в силовых алгоритмах и ускорения сходимости используется стратегия <<огрубления>>. Исходный граф $G_0$ последовательно сжимается в цепочку графов $G_1, \dots, G_k$ меньшего размера. Укладка рассчитывается для самого малого графа, а затем интерполируется обратно на исходный граф с локальной оптимизацией.

\subsection{Агрегация и упрощение визуальной сложности}

Даже при идеальной укладке отображение миллионов объектов делает визуализацию нечитаемой. Для решения этой проблемы используются методы абстракции данных.

\begin{enumerate}
	\item Кластеризация и группировка --- замена плотных групп вершин на мета-узлы (суперузлы). Это позволяет пользователю работать с графом на разных уровнях абстракции, раскрывая мета-вершины для их визуализации по мере повышения уровня детализации или, наоборот, сворачивание вершин и отображение мета-вершин взамен при уменьшении масштаба отображения графа или его видимой (рассматриваемой) части.
	\item Связывание ребер --- метод агрегации связей, при котором ребра группируются в пучки (по аналогии с кабельными трассами). Это снижает визуальный шум и позволяет выявлять высокоуровневые потоки информации между группами вершин.
	\item Топологическая фильтрация --- отображение только значимых элементов графа, например, остовного дерева или вершин с высокой метрикой центральности, с динамической подгрузкой деталей по запросу.
\end{enumerate}

\subsection{Проблематика интерактивного взаимодействия и селекции}

Интерактивность является обязательным требованием для анализа больших графов. Ключевой функцией является возможность выбора вершины курсором для получения детальной информации или выделения подграфа.

В условиях визуализации большого количества объектов (>$10^5$) задача выбора сопряжена со следующими проблемами.
\begin{itemize}
	\item Проблема точности --- при отображении всего графа размер отдельных вершин может быть меньше одного пикселя, либо вершины могут перекрывать друг друга. Стандартный клик мышью становится неэффективным.
	\item Вычислительная задержка --- линейный перебор всех объектов для определения того, какой из них находится под курсором (hit-testing), недопустим при частоте обновления экрана 60 FPS. Требуется использование пространственных индексов (квадродеревьев, k-d деревьев) или GPU-ускорения.
	\item Контекстная обратная связь --- при наведении курсора необходимо визуально выделять не только саму вершину, но и ее соседей или инцидентные ребра, что требует быстрого обхода графовых структур в памяти.
\end{itemize}

Таким образом, эффективная визуализация больших графов требует комплексного подхода, сочетающего быстрые алгоритмы укладки, методы визуальной агрегации и оптимизированные механизмы интерактивной селекции.
