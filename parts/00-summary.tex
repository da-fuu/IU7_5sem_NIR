\numberlesssection{РЕФЕРАТ}
Расчётно-пояснительная записка \pageref{LastPage}~страниц, \total{figure}~рисунков, \total{table}~таблицы, 22~источника.


\noindent
ВИЗУАЛИЗАЦИЯ ГРАФОВ, БОЛЬШИЕ ДАННЫЕ, СИЛОВЫЕ АЛГОРИТМЫ, СНИЖЕНИЕ РАЗМЕРНОСТИ, ИНТЕРАКТИВНЫЙ РЕЖИМ, ВЫБОР ВЕРШИН, КЛАСТЕРИЗАЦИЯ, НАВИГАЦИЯ, ГРАФОВЫЕ ЭМБЕДДИНГИ.

Объект исследования --- методы и алгоритмы визуализации и интерактивного взаимодействия с графовыми структурами большой размерности.

Цель работы --- проанализировать существующие методы интерактивной визуализации больших графов.
Поставленная цель достигается путём анализа предметной области, анализа и классификации 30 существующих программных решений алгоритмов укладки и способов интерактивного выбора вершин курсором.

В данной работе проведено исследование 20 методов визуализации от классических силовых алгоритмов до современных подходов на основе машинного обучения. Рассмотрены методы борьбы с информационной перегрузкой, такие как кластеризация и связывание ребер. Исследованы 10 методов навигации и выбора вершин курсором в условиях высокой плотности объектов.

В результате выполнения данной работы были классифицированы существующие методы укладки и агрегации графов по критериям: быстродействие, тип графа, устойчивость, выделение кластеров. Были классифицированы существующие методы интерактивности и навигации по критериям: точность выбора вершины, сохранение контекста, тип.
