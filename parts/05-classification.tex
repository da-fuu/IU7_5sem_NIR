\section{Сравнительный анализ существующих решений}

\subsection[Сравнительный анализ методов укладки и агрегации графов]{Сравнительный анализ методов укладки и \\ агрегации графов}

В таблице \ref{tab:layout_comparison} приведен сравнительный анализ 20 рассмотренных методов по критериям:
\begin{itemize}
	\item быстродействие --- асимптотическая временная сложность алгоритма;
	\item тип графа;
	\item устойчивость --- дает ли алгоритм схожие результаты при повторном запуске или при незначительном изменении данных;
	\item выделение кластеров --- способен ли метод визуально отделить плотные группы вершин друг от друга.
\end{itemize}

Введены следующие числовые обозначения:
\begin{itemize}
	\item устойчивость: 0 --- результат меняется при перезапуске, 1 --- результат всегда одинаков, 2 --- результат устойчив к малым изменениям, то есть при добавлении вершины граф не перестраивается целиком;
	\item выделение кластеров: 0 --- отсутствует, 1 --- неявное (кластеры видны, но не разделены), 2 --- явное (четкое визуальное разделение).
\end{itemize}

При описании быстродействия используются следующие обозначения: $N$ --- число вершин, $E$ --- число ребер, $K$ --- высота дерева (для графов, являющихся деревом).

\clearpage

\begin{tabularx}{\linewidth}{|Y|>{\centering\arraybackslash}p{2.5cm}|>{\centering\arraybackslash}p{3cm}|>{\centering\arraybackslash}p{2cm}|>{\centering\arraybackslash}p{2.5cm}|}
	\caption{\label{tab:layout_comparison}Сравнение методов укладки и агрегации}                         \\
	\hline
	Метод                          & Быстро-действие & Тип графа   & Устойчи-вость & Выделение кластеров \\ \hline
	\endfirsthead

	\multicolumn{5}{l}{{Продолжение таблицы\ \thetable{}}}                                               \\
	\hline
	Метод                          & Быстро-действие & Тип графа   & Устойчи-вость & Выделение кластеров \\ \hline
	\endhead

	\endfoot
	\endlastfoot

	Фрухтермана-Рейнгольда         & $O(N^2)$        & Любой       & 0             & 1                   \\ \hline
	Камады-Каваи                   & $O(N^3)$        & Любой       & 1             & 1                   \\ \hline
	Мажоризация напряжений         & $O(N^2)$        & Любой       & 1             & 1                   \\ \hline
	t-SNE                          & $O(N \log N)$   & Любой       & 0             & 2                   \\ \hline
	LargeVis                       & $O(N)$          & Любой       & 0             & 2                   \\ \hline
	Node2Vec                       & $O(N)$          & Любой       & 0             & 1                   \\ \hline
	VERSE                          & $O(N)$          & Любой       & 0             & 1                   \\ \hline
	Упрощенная графовая свертка    & $O(E)$          & Любой       & 1             & 1                   \\ \hline
	Иерархическое связывание ребер & $O(E \cdot K)$  & Дерево      & 1             & 2                   \\ \hline
	Топологический рыбий глаз      & $O(N \log N)$   & Любой       & 1             & 1                   \\ \hline
	FM\textsuperscript{3}          & $O(N \log N)$   & Любой       & 2             & 1                   \\ \hline
	OpenOrd                        & $O(N \log N)$   & Разреженный & 0             & 2                   \\ \hline
	LinLog                         & $O(N \log N)$   & Любой       & 0             & 2                   \\ \hline
	Размещение «Линия темы»        & $O(N)$          & Событийный  & 1             & 0                   \\ \hline
	SlashBurn                      & $O(N + E)$      & Разреженный & 1             & 2                   \\ \hline
	ForceAtlas2                    & $O(N \log N)$   & Любой       & 0             & 2                   \\ \hline
	sfdp                           & $O(N \log N)$   & Любой       & 0             & 2                   \\ \hline
	HDE                            & $O(N)$          & Любой       & 1             & 1                   \\ \hline
	Генетические алгоритмы         & $O(N^3)$        & Любой       & 0             & 1                   \\ \hline
	Гиперболическое дерево         & $O(N)$          & Дерево      & 1             & 1                   \\ \hline
\end{tabularx}



\subsection{Сравнительный анализ методов навигации и интерактивности}

В таблице \ref{tab:interaction_comparison} приведен сравнительный анализ 10 рассмотренных методов по критериям:
\begin{itemize}
	\item точность выбора вершины;
	\item сохранение контекста --- изменяется ли видимость графа;
	\item тип --- основное назначение метода.
\end{itemize}

Введены следующие числовые обозначения:
\begin{itemize}
	\item точность выбора вершины: 0 --- требует точного попадания в вершину, 1~---~требует увеличения для выбора конкретной вершины, 2 --- упрощает выбор перекрытых объектов;
	\item сохранение контекста: 0 --- видна только часть графа, 1 --- граф виден полностью, но искажен, 2 --- граф виден полностью без искажений.
\end{itemize}


\begin{tabularx}{\linewidth}{|Y|>{\centering\arraybackslash}p{2cm}|>{\centering\arraybackslash}p{2.4cm}|>{\centering\arraybackslash}p{3.8cm}|}
	\caption{\label{tab:interaction_comparison}Сравнение методов интерактивности и навигации}       \\
	\hline
	Метод                         & Точность выбора вершины & Сохранение контекста & Тип            \\ \hline
	\endfirsthead

	\multicolumn{4}{l}{{Продолжение таблицы\ \thetable{}}}                                          \\
	\hline
	Метод                         & Точность выбора вершины & Сохранение контекста & Тип            \\ \hline
	\endhead

	\endfoot
	\endlastfoot

	Семантическое масштабирование & 1                       & 0                    & Навигационный  \\ \hline
	Искажение <<Рыбий глаз>>      & 2                       & 1                    & Навигационный  \\ \hline
	Эксцентричные метки           & 2                       & 2                    & Атомарный      \\ \hline
	Инкрементальный просмотр      & 1                       & 0                    & Навигационный  \\ \hline
	Связывание и кисть            & 1                       & 2                    & Групповой      \\ \hline
	Волшебные линзы               & 2                       & 1                    & Фильтрационный \\ \hline
	Лассо-выделение               & 1                       & 2                    & Групповой      \\ \hline
	Приближение и переход         & 2                       & 1                    & Навигационный  \\ \hline
	Метафора <<Резиновый жгут>>   & 2                       & 1                    & Навигационный  \\ \hline
	Динамические запросы          & 2                       & 0                    & Фильтрационный \\ \hline
\end{tabularx}
