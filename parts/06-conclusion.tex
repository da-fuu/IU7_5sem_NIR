\numberlesssection{ЗАКЛЮЧЕНИЕ}

В данной работе проведено исследование методов 20 визуализации от классических силовых алгоритмов до современных подходов на основе машинного обучения. Рассмотрены методы борьбы с информационной перегрузкой, такие как кластеризация и связывание ребер. Исследованы 10 методов навигации и выбора вершин курсором в условиях высокой плотности объектов.

Цель работы достигнута --- были проанализированы существующие методы интерактивной визуализации больших графов. Были выполнены следующие задачи:
\begin{itemize}
	\item проведён анализ предметной области, введены базовые определения области визуализации графов и обозначены ключевые проблемы отображения больших массивов данных;
	\item проанализированы 30 существующих программных решений алгоритмов укладки и способов интерактивного выбора вершин курсором;
	\item сформулированы условия и критерии сравнения найденных программных решений;
	\item построена сравнительная таблица для найденных решений на основе сформулированных условий и критериев.
\end{itemize}
