\section{Обзор соответствующих программных решений}

\subsection[Методы пространственной укладки и агрегации графов]{Методы пространственной укладки и \\ агрегации графов}

В данном разделе рассматриваются ключевые алгоритмы размещения вершин графа на плоскости.

\subsubsection{Алгоритм Фрухтермана --- Рейнгольда}
Это классический силовой алгоритм, который моделирует граф как физическую систему. Вершины рассматриваются как атомные частицы или стальные кольца, которые отталкивают друг друга, а ребра --- как пружины, притягивающие связанные вершины. Цель алгоритма --- минимизировать энергию системы, обеспечивая равномерное распределение вершин и минимизацию пересечений ребер~\cite{gibson2013}.

Основные этапы работы следующие.
\begin{enumerate}
	\item Инициализация --- вершины размещаются случайным образом в ограниченной области.
	\item Расчет сил отталкивания --- между каждой парой вершин действует сила $f_{rep}(d) = k^2 / d$, где $d$ --- текущее расстояние, а $k$ --- оптимальное расстояние.
	\item Расчет сил притяжения --- только для связанных вершин рассчитывается сила $f_{attr}(d) = d^2 / k$.
	\item Смещение вершин --- каждая вершина сдвигается в направлении вектора результирующей силы, но не дальше, чем позволяет текущая <<температура>> (параметр, ограничивающий максимальное смещение).
	\item Охлаждение --- температура постепенно снижается с каждой итерацией до полной остановки системы.
\end{enumerate}

\subsubsection{Алгоритм Камады --- Каваи}
Данный алгоритм относится к энергетическим алгоритмам. В его основе лежит идея, что геометрическое расстояние между вершинами на экране должно быть пропорционально их теоретико-графовому расстоянию (длине кратчайшего пути). Алгоритм ищет конфигурацию с минимальной общей энергией <<напряжения>> пружин~\cite{gansner2004}.

Основные этапы работы следующие.
\begin{enumerate}
	\item Вычисление матрицы кратчайших путей $d_{ij}$ для всех пар вершин, обычно с помощью алгоритма Флойда --- Уоршелла или BFS.
	\item Определение идеальной длины пружины $l_{ij} = L \cdot d_{ij}$ и ее жесткости $k_{ij} = K / d_{ij}^2$.
	\item Минимизация функции энергии $E = \sum_{i<j} k_{ij} (|p_i - p_j| - l_{ij})^2$, где $p_i$ и $p_j$ --- координаты вершин.
	\item Оптимизация --- на каждом шаге выбирается вершина с наибольшей энергией смещения и перемещается в позицию локального минимума с помощью метода Ньютона --- Рафсона.
\end{enumerate}

Пример применения алгоритма Камады --- Каваи представлен на рисунке~\ref{img:stress_KK.png}.

\jimg{stress_KK.png}{Пример применения алгоритма Камады --- Каваи (справа) и мажоризации напряжений (слева) к одному исходному графу~\cite{gansner2004}}

\subsubsection{Метод мажоризации напряжений}
Этот метод является современной оптимизацией алгоритма Камады~---~Каваи. Вместо локального перемещения по одной вершине, он использует глобальную оптимизацию функции стресса. Мажоризация позволяет заменить сложную целевую функцию на более простую вспомогательную функцию (мажоранту), минимум которой найти легче~\cite{gansner2004}.

Основные этапы работы следующие.
\begin{enumerate}
	\item Вычисление функции стресса по формуле~\eqref{eq:majorization}.
	      \begin{equation}
		      stress(X) = \sum_{i<j} w_{ij} (||X_i - X_j|| - d_{ij})^2.
		      \label{eq:majorization}
	      \end{equation}
	\item Построение мажорирующей функции, которая является квадратичной формой и касается исходной функции в текущей точке.
	\item Решение системы линейных уравнений вида $L^w X = L^Z Z$, где $L^w$~---~взвешенный Лапласиан (дифференциальный оператор, эквивалентый последовательному взятию операций градиента и дивергенции), для нахождения минимума мажоранты.
	\item Итеративное повторение процесса до сходимости. Метод гарантирует монотонное уменьшение энергии и является более стабильным, чем метод Ньютона --- Рафсона.
\end{enumerate}

Пример применения алгоритма мажоризации напряжений представлен на рисунке~\ref{img:stress_KK.png}.

\subsubsection{Метод t-SNE}
Метод нелинейного снижения размерности, который часто используется для визуализации графовых эмбеддингов или матриц смежности. Алгоритм t-SNE преобразует расстояния в многомерном пространстве в условные вероятности, стремясь сохранить локальную структуру данных~\cite{tang2016}.

Основные этапы работы следующие.
\begin{enumerate}
	\item Вычисление попарных сходств в исходном пространстве высокой размерности с использованием распределения Гаусса.
	\item Определение сходств в низкоразмерном (2D) пространстве с использованием t-распределения Стьюдента с одной степенью свободы (чтобы избежать проблемы <<скученности>>). Вероятность $q_{ij}$ определяется по формуле~\eqref{eq:t-sne}.
	      \begin{equation}
		      q_{ij} = \frac{(1 + ||y_i - y_j||^2)^{-1}}{\sum_{k \neq l} (1 + ||y_k - y_l||^2)^{-1}}
		      \label{eq:t-sne}.
	      \end{equation}
	\item Минимизация дивергенции Кульбака --- Лейблера между распределениями исходного и целевого пространств с помощью градиентного спуска.
\end{enumerate}

\subsubsection{Метод LargeVis}
Метод, разработанный для визуализации очень больших графов и многомерных данных (масштаба миллионов вершин). Он устраняет недостатки t-SNE (высокую вычислительную сложность) путем аппроксимации графа ближайших соседей и использования вероятностной модели~\cite{tang2016}.

Основные этапы работы следующие.
\begin{enumerate}
	\item Построение приближенного графа K-ближайших соседей (KNN-граф). Используются деревья случайной проекции (метод разбиения пространства случайными гиперплоскостями для быстрого поиска соседей в многомерных данных, основанный на k-d деревьях~\cite{tang2016}) для быстрого разбиения пространства.
	\item Уточнение графа с помощью алгоритма распространения соседей, основанного на принципе <<сосед моего соседа --- вероятно, мой сосед>>.
	\item Визуализация графа с использованием вероятностной модели, которая максимизирует вероятность наблюдения связей. Оптимизация выполняется с помощью асинхронного стохастического градиентного спуска, что обеспечивает линейную сложность $O(N)$.
\end{enumerate}

\img{15cm}{largeviz.png}{Пример применения алгоритма LargeVis~\cite{habrKovalev}}

\subsubsection{Метод Node2Vec}
Метод обучения признаков, который преобразует вершины графа в векторы низкой размерности. Хотя сам по себе Node2Vec не дает координат на плоскости, полученные векторы затем визуализируются методами снижения размерности (например, t-SNE или PCA). Основная идея --- использование смещенных случайных блужданий~\cite{grover2016}.

Основные этапы работы следующие.
\begin{enumerate}
	\item Генерация случайных блужданий фиксированной длины из каждой вершины. Стратегия обхода регулируется параметрами $p$ (вероятность возврата) и $q$ (вероятность ухода вглубь), что позволяет балансировать между BFS и DFS.
	\item Обучение модели Skip-gram (аналогично Word2Vec), где <<предложениями>> являются последовательности вершин в блужданиях. Максимизируется вероятность появления соседей в контексте данной вершины, которая вычисляется по формуле~\eqref{eq:n2v}.
	      \begin{equation}
		      P = \max_f \sum_{u \in V} \log Pr(N_S(u) | f(u)).
		      \label{eq:n2v}
	      \end{equation}
	\item Получение векторов признаков $f(u)$ и их последующая проекция в 2D.
\end{enumerate}

Пример применения алгоритма Node2Vec представлен на рисунке~\ref{img:n2v.png}.

\img{15cm}{n2v.png}{Пример применения алгоритма Node2Vec~\cite{habrKovalev}}

\subsubsection{Метод VERSE}
Метод VERSE (Versatile Graph Embeddings from Similarity Measures) представляет собой универсальный фреймворк для получения эмбеддингов (компактных векторных представлений вершин в низкоразмерном пространстве, сохраняющих структурную информацию исходного графа). В отличие от алгоритмов, использующих фиксированные стратегии случайных блужданий (как Node2Vec), данный метод не привязан к жесткой схеме сэмплирования, а обучается явно сохранять распределение любой выбранной пользователем меры сходства вершин. На вход алгоритм принимает структуру графа и выбранную функцию сходства, а результатом работы является матрица признаков, строки которой служат координатами вершин в латентном пространстве для задач визуализации, классификации или предсказания связей~\cite{tsitsulin2018}.

Ключевой особенностью подхода является использование мер сходства --- функций, которые определяют степень близости или родства между любой парой вершин в графе. Рассматриваются три основные меры: Personalized PageRank, которая учитывает глобальную структуру и пути в графе; SimRank, основанная на структурной эквивалентности; и простая смежность, учитывающая только непосредственных соседей. Выбор меры позволяет адаптировать метод под конкретную прикладную задачу.

Основные этапы работы следующие.
\begin{enumerate}
	\item Выбор меры сходства $sim_G$, которая для каждой вершины задает эталонное распределение вероятностей перехода к другим вершинам графа.
	\item Обучение однослойной нейронной сети, задача которой --- сформировать векторы вершин таким образом, чтобы их скалярное произведение в пространстве эмбеддингов аппроксимировало выбранную меру сходства.
	\item Оптимизация теоретической целевой функции, представляющей собой дивергенцию Кульбака --- Лейблера. Эта функция минимизирует различие между распределением сходства в исходном графе ($sim_G$) и восстановленным распределением в пространстве эмбеддингов. Иными словами, алгоритм <<штрафует>> модель, если «расстояния» между векторами не соответствуют «расстояниям» в графе.
	\item Применение техники Noise Contrastive Estimation для масштабируемого обучения. Поскольку явное вычисление полной матрицы сходства и минимизация KL-дивергенции вычислительно слишком дороги, задача сводится к обучению логистической регрессии. Модель учится отличать примеры вершин, выбранные из реального распределения сходства, от случайных <<шумовых>> вершин. Параметры обновляются с помощью стохастического градиентного спуска, что позволяет обрабатывать графы с миллионами узлов.
\end{enumerate}

Пример применения алгоритма VERSE представлен на рисунке~\ref{img:verse.png}.

\img{11cm}{verse.png}{Пример применения алгоритма VERSE~\cite{habrKovalev}}

\subsubsection{Упрощенная графовая свертка}
Упрощенная модель графовых сверточных сетей, которая устраняет нелинейности между слоями, сводя глубокую сеть к линейной модели. Это значительно ускоряет обучение и делает модель интерпретируемой, действуя как низкочастотный фильтр в спектральной области~\cite{wu2019}.

Основные этапы работы следующие.
\begin{enumerate}
	\item Добавление петель к графу (ренормализация): $\tilde{A} = A + I$.
	\item Расчет нормализованной матрицы смежности $\tilde{S} = \tilde{D}^{-1/2}\tilde{A}\tilde{D}^{-1/2}$.
	\item Распространение признаков (feature propagation) путем $K$-кратного умножения матрицы признаков $X$ на $\tilde{S}$ по формуле~\eqref{eq:sgc}.
	      \begin{equation}
		      \bar{X} = \tilde{S}^K X.
		      \label{eq:sgc}
	      \end{equation}
	\item Использование полученных сглаженных признаков $\bar{X}$ для классификации или визуализации (например, через PCA).
\end{enumerate}

\subsubsection{Иерархическое связывание ребер}
Метод, предназначенный для решения проблемы визуального шума  в плотных графах путем группировки смежных ребер. Он использует иерархическую структуру данных (дерево) для направления ребер~\cite{holten2006}.

Основные этапы работы следующие.
\begin{enumerate}
	\item Построение иерархии (если она не задана, может использоваться алгоритм кластеризации).
	\item Определение пути для каждого ребра графа: ребро между вершинами $P$ и $Q$ строится вдоль пути в иерархии от $P$ до $Q$ через их наименьшего общего предка.
	\item Моделирование ребра как B-сплайна, где контрольные точки определяются узлами иерархии.
	\item Регулировка степени натяжения параметром $\beta$. При $\beta=0$ ребра прямые, при $\beta=1$ они сильно прижаты к иерархическому дереву, образуя пучки.
\end{enumerate}

\subsubsection{Топологический рыбий глаз}
Метод навигации и визуализации очень больших графов, основанный на гибридном представлении. В отличие от геометрического искажения, здесь используется топологическое упрощение удаленных областей~\cite{gansner2005}.

Основные этапы работы следующие.
\begin{enumerate}
	\item Предварительное вычисление иерархии огрубленных графов путем последовательного стягивания вершин.
	\item Выбор пользователем фокусной вершины.
	\item Построение <<гибридного графа>> на лету: окрестность фокуса берется из исходного детального графа, а удаленные регионы заменяются соответствующими частями из огрубленных уровней иерархии.
	\item Применение геометрического искажения к полученному гибридному графу для обеспечения плавного перехода между уровнями детализации.
\end{enumerate}

Пример применения метода топологический рыбий глаз представлен на рисунке~\ref{img:topological.png}.

\jimg{topological.png}{Пример применения метода топологический рыбий глаз с центром в различных местах графа~\cite{gansner2005}}

\subsubsection{Алгоритм FM\textsuperscript{3}}
Данный алгоритм представляет собой многоуровневый подход к силовой укладке, специально разработанный для визуализации больших графов. Он решает проблему медленной сходимости и попадания в локальные минимумы, характерную для классических силовых методов при большом количестве вершин~\cite{vonLandesberger2011}.

Основные этапы работы следующие.
\begin{enumerate}
	\item Огрубление --- исходный граф последовательно сжимается путем объединения смежных вершин в супер-узлы, создавая иерархию графов уменьшающегося размера.
	\item Укладка на нижнем уровне --- для самого маленького графа в иерархии рассчитывается начальная укладка с использованием силового алгоритма.
	\item Интерполяция и уточнение --- координаты вершин переносятся с грубого уровня на более детальный. На каждом уровне применяется локальная силовая оптимизация (релаксация) для уточнения позиций. Для ускорения расчета сил отталкивания используется аппроксимация с помощью мультипольного метода (аналогично алгоритму Барнса --- Хата).
\end{enumerate}

\subsubsection{Алгоритм OpenOrd}
Алгоритм укладки, основанный на силовом подходе, но модифицированный для выявления кластерной структуры в очень больших графах (до миллионов узлов). Ключевой особенностью является агрессивное отсечение длинных ребер в процессе оптимизации~\cite{habrKovalev}.

Основные этапы работы следующие.
\begin{enumerate}
	\item Жидкая фаза: допускаются большие перемещения вершин, высокая температура симуляции.
	\item Фаза расширения: граф растягивается для улучшения разделимости.
	\item Фаза охлаждения и отсечения: температура снижается, и длинные ребра, соединяющие удаленные кластеры, игнорируются при расчете сил притяжения. Это позволяет кластерам обособиться друг от друга.
	\item Завершающая фаза: локальная доводка позиций для улучшения эстетического вида внутри кластеров.
\end{enumerate}

\subsubsection{Метод LinLog}
Энергетическая модель укладки, предложенная Ноаком как альтернатива модели Фрухтермана --- Рейнгольда для лучшего выявления кластеров. Название происходит от типа используемых сил: сила притяжения логарифмическая, а сила отталкивания --- линейная (в некоторых интерпретациях, наоборот, зависит от контекста формулировки энергии или силы)~\cite{apanovich2019}.

Основные этапы работы следующие.
\begin{enumerate}
	\item Определение целевой функции энергии, где штраф за длину ребра растет линейно (что сильно стягивает удаленные связанные узлы), а отталкивание препятствует коллапсу. В терминах энергии минимизируется выражение~\eqref{eq:linlog}.
	      \begin{equation}
		      E = \sum_{(u,v) \in E} ||p_u - p_v|| - \sum_{(u,v) \in V \times V} \ln ||p_u - p_v||.
		      \label{eq:linlog}
	      \end{equation}
	\item Оптимизация этой функции приводит к тому, что узлы с высокой плотностью связей (кластеры) группируются очень тесно, а расстояние между кластерами становится значительным, что делает структуру сообществ визуально очевидной.
\end{enumerate}

\subsubsection{Размещение <<Линия темы>>}
Специализированный алгоритм укладки, используемый для анализа социальных сетей или событийных графов, где важно выделить ключевые объекты (<<темы>>) и их связи с окружением. Метод эффективен для визуализации хронологии или иерархии влияния~\cite{kolomeychenko2014}.

Основные этапы работы следующие.
\begin{enumerate}
	\item Выбор ключевых вершин (тем), которые будут размещены на базовой горизонтальной линии.
	\item Размещение линии темы: ключевые вершины выстраиваются в линию, длина которой может варьироваться.
	\item Размещение окружения: остальные вершины, связанные с темой, размещаются перпендикулярно линии (сверху или снизу).
	\item Маршрутизация ребер: связи между второстепенными вершинами и линией темы рисуются вертикально, минимизируя пересечения с другими элементами.
\end{enumerate}

Пример размещения <<Линия темы>> представлен на рисунке~\ref{img:line.png}.

\jimg{line.png}{Пример размещения <<Линия темы>>}

\subsubsection{Матричное упорядочение SlashBurn}
Метод, предназначенный для визуализации и сжатия матриц смежности больших разреженных графов реального мира (социальные сети, веб-графы). Он использует свойство степенного распределения степеней вершин для оптимизации порядка строк и столбцов матрицы~\cite{apanovich2019}.

Основные этапы работы следующие.
\begin{enumerate}
	\item Идентификация хабов: поиск вершин с наивысшей степенью.
	\item Удаление хабов: вершины-хабы и инцидентные им ребра временно удаляются из графа.
	\item Выделение компонент связности: оставшийся граф распадается на множество мелких компонент.
	\item Переупорядочивание: хабы помещаются в начало матрицы, а вершины компонент связности группируются вместе. Это создает характерную форму <<стрелы>> в матрице смежности, концентрируя ненулевые элементы и разрежая остальное пространство.
\end{enumerate}

\subsubsection{Алгоритм ForceAtlas2}
Специализированный силовой алгоритм, разработанный для визуализации сетей <<малого мира>> и безмасштабных графов. Является алгоритмом по умолчанию в популярном инструменте Gephi. Он отличается линейно-логарифмической моделью взаимодействия (сила притяжения пропорциональна расстоянию, отталкивания — обратна расстоянию) и адаптивной скоростью сходимости, что позволяет пользователю интерактивно наблюдать за процессом укладки \cite{jacomy2014}.

Основные этапы работы следующие.
\begin{enumerate}
	\item Расчет сил: отталкивание между всеми узлами (оптимизировано через Барнса --- Хата, $O(N \log N)$) и притяжение только между соседями (линейно от расстояния).
	\item Учет весов ребер и степени вершин: узлы с высокой степенью отталкиваются сильнее, что выталкивает хабы на периферию кластеров.
	\item Адаптивное охлаждение: шаг смещения зависит от локальной <<температуры>>. Если узел колеблется, шаг уменьшается; если движется в одном направлении --- увеличивается.
	\item Предотвращение перекрытий: дополнительная стадия для разведения узлов на расстояние их радиуса.
\end{enumerate}

Пример применения алгоритма ForceAtlas2 представлен на рисунке~\ref{img:fa2.png}.

\img{13cm}{fa2.png}{Пример применения алгоритма ForceAtlas2~\cite{habrKovalev}}

\subsubsection{Алгоритм sfdp}
Многоуровневый алгоритм силовой укладки, входящий в пакет Graphviz. Предназначен для визуализации очень больших графов (миллионы узлов). Основная идея --- использование огрубления графа для быстрого нахождения глобальной структуры и последующая детализация. Алгоритм эффективно использует память и процессорное время \cite{hu2005}.

Основные этапы работы следующие.
\begin{enumerate}
	\item Построение иерархии: исходный граф сворачивается в серию уменьшающихся графов.
	\item Начальная укладка: самый грубый граф укладывается силовым методом.
	\item Уточнение: координаты переносятся на более детальный уровень. Применяется алгоритм укладки с использованием аппроксимации дальнодействующих сил для уточнения позиций.
	\item Адаптация региона: алгоритм может адаптировать плотность укладки в разных регионах графа, чтобы избежать наложений в плотных кластерах.
\end{enumerate}

\subsubsection{Метод HDE}
Метод быстрого размещения графа, основанный на проецировании из пространства высокой размерности. Он отличается высокой скоростью работы, что делает его пригодным для интерактивного анализа графов с сотнями тысяч вершин~\cite{gibson2013}.

Основные этапы работы следующие.
\begin{enumerate}
	\item Выбор опорных вершин: выбирается небольшое множество вершин (обычно 50–100).
	\item Встраивание в многомерное пространство: для каждой вершины графа вычисляется вектор расстояний (кратчайших путей) до всех опорных вершин. Если выбрано $K$ опорных точек, каждая вершина получает координаты в $K$-мерном пространстве.
	\item Снижение размерности: применяется метод главных компонент (PCA) к полученной матрице координат размера $N \times K$ для проецирования данных на 2D плоскость таким образом, чтобы сохранить максимальную дисперсию.
\end{enumerate}

\subsubsection{Генетические алгоритмы укладки}
Подход к размещению графа, основанный на принципах эволюционной биологии. Он позволяет находить глобально оптимальные укладки, учитывая сложные и конфликтующие эстетические критерии, которые трудно формализовать в виде гладкой функции энергии~\cite{kostina2014}.

Основные этапы работы следующие.
\begin{enumerate}
	\item Создание начальной популяции: генерируется множество случайных вариантов раскладки графа.
	\item Оценка приспособленности: для каждого варианта вычисляется оценка качества, учитывающая количество пересечений ребер, равномерность распределения вершин, длину ребер и другие критерии.
	\item Селекция и скрещивание: отбираются лучшие варианты, и их координаты комбинируются для создания новых решений.
	\item Мутация: вносятся случайные изменения в координаты вершин для предотвращения преждевременной сходимости. Процесс повторяется до достижения критерия остановки.
\end{enumerate}

\subsubsection{Гиперболическое дерево}
Метод визуализации иерархических графов (деревьев), использующий неевклидову геометрию. Он реализует концепцию <<фокус + контекст>>, позволяя отображать большие структуры данных в ограниченном пространстве~\cite{kasyanov2013}.

Основные этапы работы следующие.
\begin{enumerate}
	\item Размещение в гиперболической плоскости: узлы дерева укладываются на гиперболической плоскости (обычно используется модель диска Пуанкаре). Корневой узел помещается в центр.
	\item Выделение пространства: каждому узлу выделяется сектор, размер которого экспоненциально уменьшается с удалением от центра. Это свойство гиперболической геометрии позволяет вместить экспоненциально растущее число узлов.
	\item Проекция на экран: гиперболические координаты преобразуются в экранные евклидовы координаты. Узлы в центре (фокусе) отображаются крупно, а удаленные узлы сжимаются у границ диска.
\end{enumerate}


\subsection[Методы интерактивного взаимодействия и выбора вершин]{Методы интерактивного взаимодействия и \\ выбора вершин}

Для эффективной работы с большими графами критически важны методы, позволяющие пользователю точно выбирать элементы (вершины и ребра) даже в условиях высокой плотности и перекрытия объектов.

\subsubsection{Семантическое масштабирование}
Метод взаимодействия, при котором изменение масштаба (зумирование) влияет не только на геометрический размер объектов, но и на способ их представления. При отдалении вершины могут отображаться как точки или агрегированные кластеры, а при приближении курсора к интересующей области детализация повышается: появляются метки, иконки и внутренние связи. Это позволяет пользователю выбирать вершину только тогда, когда она достаточно велика и различима~\cite{yi2007}.

\subsubsection{Искажение <<Рыбий глаз>>}
Техника искажения экранного пространства, основанная на метафоре широкоугольной линзы. Область под курсором мыши увеличивается, раздвигая соседние элементы, в то время как периферия сжимается. Это позволяет пользователю точно выбрать мелкую вершину в плотном кластере, не теряя при этом глобального контекста графа и связи с окружающими узлами~\cite{apanovich2008}.

\subsubsection{Эксцентричные метки}
Специализированный метод выбора и идентификации вершин в очень плотных областях, где стандартные текстовые подписи перекрывают друг друга. При наведении курсора на скопление узлов система определяет окрестность фокуса и выводит список меток для всех попавших в нее вершин в виде упорядоченного списка рядом с курсором. Линии-выноски соединяют пункты списка с соответствующими узлами, позволяя точно выбрать нужный объект~\cite{vonLandesberger2011}.

\subsubsection{Инкрементальный просмотр}
Подход к навигации, при котором изначально пользователю показывается лишь небольшая часть графа (например, корневой узел или выборка важных вершин). Остальная часть графа скрыта. При выборе вершины курсором система динамически подгружает и отображает ее соседей. Это позволяет работать с графами бесконечного размера, избегая перегрузки экрана и упрощая выбор следующего узла для перехода~\cite{kasyanov2013}.

\subsubsection{Связывание и кисть}
Метод группового выбора, используемый при наличии нескольких представлений данных (например, граф и матрица смежности). Пользователь выделяет группу вершин курсором (<<кистью>>) в одном окне (например, обводя область на графе), и соответствующие этим вершинам элементы автоматически подсвечиваются во всех остальных представлениях. Это облегчает идентификацию и выбор вершин, которые могут быть скрыты или труднодоступны в одном из визуальных представлений~\cite{yi2007}.

\subsubsection{Волшебные линзы}
Интерактивный инструмент в виде подвижной области (линзы), перемещаемой курсором поверх графа. Внутри линзы меняется способ отображения данных: например, могут скрываться ребра для облегчения выбора вершин, изменяться цветовая схема или показываться скрытые атрибуты. Это позволяет производить селекцию в локальном контексте без изменения глобальных настроек визуализации~\cite{vonLandesberger2011}.

\subsubsection{Лассо-выделение}
Инструмент для выбора множества вершин произвольной формы. Пользователь <<рисует>> курсором замкнутый контур вокруг интересующей группы узлов. Все вершины, попавшие внутрь контура, становятся выбранными. Этот метод значительно эффективнее прямоугольного выделения для графов со сложной геометрической структурой или при необходимости выделить конкретный кластер неправильной формы~\cite{vonLandesberger2011}.

\subsubsection{Приближение и переход}
Техника навигации для выбора соседей вершины. При выборе узла его непосредственные соседи, которые могут находиться далеко на экране, плавно перемещаются (<<подтягиваются>>) ближе к выбранному узлу. Это позволяет пользователю легко выбрать следующую вершину для перехода без необходимости прокручивать экран и искать связи в большом количестве ребер~\cite{vonLandesberger2011}.

\subsubsection{Метафора <<Резиновый жгут>>}
Метод геометрической трансформации, позволяющий пользователю растягивать область просмотра, словно резиновый лист. Захватив курсором точку на холсте и потянув ее, пользователь растягивает прилегающую область графа, увеличивая расстояние между вершинами. Это позволяет временно <<разрядить>> плотный участок графа для удобного выбора конкретной вершины без потери топологии~\cite{apanovich2008}.

\subsubsection{Динамические запросы}
Метод непрямого выбора вершин через элементы интерфейса (слайдеры, ползунки, чекбоксы). Пользователь фильтрует вершины по атрибутам (например, <<показать вершины со степенью > 100>>). Отфильтрованные вершины либо скрываются, либо затеняются, оставляя на экране только те объекты, которые соответствуют критериям. Это позволяет легко выбирать целевые вершины курсором, убрав визуальный шум~\cite{shneiderman1996}.
