\numberlesssection{ВВЕДЕНИЕ}

Графы являются универсальным языком для описания сложных систем взаимосвязанных объектов в самых различных предметных областях: от социальных сетей и биоинформатики до транспортных систем и компьютерных сетей. Визуализация таких структур играет ключевую роль в анализе данных, позволяя исследователям выявлять скрытые паттерны, кластеры и аномалии, которые невозможно обнаружить при изучении табличных данных или статистических метрик.

С наступлением эры больших данных размерность анализируемых графов выросла до десятков тысяч и миллионов вершин. В этих условиях классические подходы к визуализации сталкиваются с двумя фундаментальными проблемами. Первая — алгоритмическая и вычислительная сложность: отрисовка и пересчет укладки в реальном времени требуют значительных ресурсов. Вторая — проблема когнитивного восприятия: плотные графы превращаются в нечитаемый <<волосяной шар>>, где невозможно различить отдельные элементы.

В контексте работы с большими графами статические изображения теряют свою информативность. Критически важным становится интерактивный режим, обеспечивающий навигацию, фильтрацию и детализацию по требованию. Одной из центральных задач интерфейса в таких системах является предоставление функциональности выбора вершины курсором. При отображении миллионов объектов задача быстрого определения элемента под курсором и визуального отклика системы (подсветки, вывода информации) становится нетривиальной технической проблемой, требующей применения пространственных индексов, GPU-ускорения и специализированных алгоритмов.

Цель данной научно-исследовательской работы --- проанализировать существующие методы интерактивной визуализации больших графов. Для достижения поставленной цели необходимо выполнить следующие задачи:
\begin{itemize}
	\item провести анализ предметной области, ввести базовые определения области визуализации графов и обозначить ключевые проблемы отображения больших массивов данных;
	\item проанализировать 30 существующих программных решений алгоритмов укладки и способов интерактивного выбора вершин курсором;
	\item сформулировать условия и критерии сравнения найденных программных решений;
	\item построить сравнительную таблицу для найденных решений на основе сформулированных условий и критериев.
\end{itemize}
